\documentclass[a4paper,12pt]{article}
\usepackage[utf8]{inputenc}     % Umlaute
\usepackage[T1]{fontenc}        % Korrekte Trennung
\usepackage[ngerman]{babel}     % Deutsche Spracheinstellungen
\usepackage{amsmath,amssymb}    % Mathematische Symbole
\usepackage{booktabs}           % Für schönere Tabellenlinien
\usepackage{array}              % Bessere Tabellenoptionen
\usepackage{geometry}           % Seitenränder anpassen
\geometry{left=2.5cm,right=2.5cm,top=2.5cm,bottom=2.5cm}

\setlength{\parindent}{0pt}     % Kein Einrücken nach Absätzen
\setlength{\parskip}{1ex}       % Kleiner Abstand zwischen Absätzen

\begin{document}

\section*{Lösungsideen und Bearbeitungshinweise zu den Aufgaben}

\textbf{Aufgabe 1a}

\emph{Finde heraus, nach welchem System die Zahlen dargestellt werden und wie viele Werte mit den vier Leuchten insgesamt beschrieben werden können. Beantworte außerdem:}
\begin{enumerate}
  \item \emph{Was ist die kleinste/größte Zahl, die angezeigt werden kann?}
    \item \emph{Überlege dir für jede Lampe: Welche Zahl wird addiert, wenn diese Lampe eingeschaltet wird und wie hängen diese Zahlen zusammen?}
    \end{enumerate}

    \subsection*{Erklärung}

    \paragraph{1.\ Darstellungssystem}
    Das beschriebene System ist das \textbf{Binärsystem} (Zweiersystem). Jede Lampe entspricht einem Wert, der einer \textbf{Zweierpotenz} zugeordnet ist.

    \begin{itemize}
     \item Bei genau 4 Lampen (von rechts nach links gelesen) haben wir typischerweise die Werte:
      \[
         \text{Lampe 1} : 2^0 = 1,\quad
            \text{Lampe 2} : 2^1 = 2,\quad
               \text{Lampe 3} : 2^2 = 4,\quad
                  \text{Lampe 4} : 2^3 = 8.
                   \]
                    \item Ob Lampe 1 rechts oder links steht, ist nicht entscheidend. Wichtig ist nur, dass jede Lampe einem festgelegten Zweierpotenz-Wert entspricht.
                    \end{itemize}

                    \paragraph{2.\ Kleinste und größte Zahl}
                    \begin{itemize}
                     \item \textbf{Kleinste Zahl}: 0 (wenn alle Lampen aus sind).
                      \item \textbf{Größte Zahl}: 15 (wenn alle 4 Lampen an sind). Denn
                       \[
                          8 + 4 + 2 + 1 = 15.
                           \]
                           \end{itemize}

                           \paragraph{3.\ Welche Zahl wird addiert, wenn eine Lampe eingeschaltet wird?}
                           \begin{itemize}
                            \item Lampe 1 (ganz rechts): \(+1\)
                             \item Lampe 2: \(+2\)
                              \item Lampe 3: \(+4\)
                               \item Lampe 4: \(+8\)
                               \end{itemize}
                               Die Summe aller \emph{eingeschalteten} Lampen ergibt die angezeigte Dezimalzahl.

                               \paragraph{4.\ Wie viele Werte können mit 4 Lampen insgesamt dargestellt werden?}
                               Mit 4 Binärstellen (4 Lampen) kann man \(\,2^4 = 16\) verschiedene Werte abbilden, nämlich 0 bis 15.

                               \bigskip

                               \textbf{Aufgabe 1b}

                               \emph{Folgere, wie die Zahlen 32 und 47 nach diesem System dargestellt werden müssten, und gib beide Darstellungen an. (Hinweis: Die Zahl der Leuchten wird erweitert.)}

                               Da 4 Lampen nur bis 15 reichen, benötigen wir mehr ``Binärstellen'' (Lampen), um 32 bzw.\ 47 darzustellen.

                               \begin{itemize}
                                \item Für den Bereich bis mindestens 47 braucht man \textbf{6 Lampen}, denn mit 5 Lampen kommt man nur bis:
                                   \[
                                        2^5 - 1 = 31,
                                           \]
                                              und das reicht für 32/47 nicht aus.
                                              \end{itemize}

                                              \paragraph{Zahl 32}
                                              \[
                                                32_{\,(10)} = 2^5 = 100000_2
                                                \]
                                                In einer 6-Bit-Darstellung wäre also nur die Lampe für \(2^5\) an, alle anderen (16, 8, 4, 2, 1) sind aus.

                                                \paragraph{Zahl 47}
                                                \[
                                                  47 = 32 + 8 + 4 + 2 + 1 = 101111_2
                                                  \]
                                                  - Lampe 6 (32): \emph{an} \\
                                                  - Lampe 5 (16): \emph{aus} \\
                                                  - Lampe 4 (8): \emph{an} \\
                                                  - Lampe 3 (4): \emph{an} \\
                                                  - Lampe 2 (2): \emph{an} \\
                                                  - Lampe 1 (1): \emph{an}

                                                  \bigskip

                                                  \textbf{Aufgabe 2}

                                                  \emph{Lies dir die entsprechenden Buchseiten durch und erstelle dir selbst eine Übersicht über die Codierung im Binärsystem.}

                                                  Hier erstellst du dir ein eigenes kleines ``Merkblatt'' zum Binärsystem:
                                                  \begin{itemize}
                                                   \item Jede Stelle (Bit) steht für eine Zweierpotenz \(2^k\), beginnend bei \(2^0\) ganz rechts, dann \(2^1\), \(2^2\), usw.
                                                    \item Beispiel:
                                                       \[
                                                            5 = 4 + 1 = 101_2 \quad\text{(in einer 3-Bit-Darstellung)},
                                                               \]
                                                                  \[
                                                                       13 = 8 + 4 + 1 = 1101_2 \quad\text{(in einer 4-Bit-Darstellung)}.
                                                                          \]
                                                                           \item Die maximal darstellbare Zahl mit \(n\) Bits ist \(\,2^n - 1\).
                                                                           \end{itemize}

                                                                           Eine Beispiel-Tabelle:

                                                                           \begin{center}
                                                                           \begin{tabular}{@{}rcc@{}}
                                                                           \toprule
                                                                           \textbf{Dezimal} & \textbf{Binär} & \textbf{Potenzen} \\
                                                                           \midrule
                                                                           0 & 0000 & -- \\
                                                                           1 & 0001 & \(2^0\) \\
                                                                           2 & 0010 & \(2^1\) \\
                                                                           3 & 0011 & \(2^1 + 2^0\) \\
                                                                           4 & 0100 & \(2^2\) \\
                                                                           \vdots & \vdots & \vdots \\
                                                                           \bottomrule
                                                                           \end{tabular}
                                                                           \end{center}

                                                                           \bigskip

                                                                           \textbf{Aufgabe 3 (B.S.\ 53/2)}

                                                                           \emph{„Zum Geburtstag viel Glück – und wenige Kerzen“: Ermitteln Sie, nach welchem Prinzip Minas Gäste ihr Alter dargestellt haben, beantworten Sie Minas Fragen und diskutieren Sie, woran man erkennt, wer älter ist. Formulieren Sie anschließend eine allgemeine Regel.}

                                                                           \subsection*{Grundprinzip (Kerzensystem)}
                                                                           Das Alter wird im \textbf{Binärsystem} durch Kerzen abgebildet. Jede Kerze steht für eine Zweierpotenz:
                                                                           \[
                                                                             2^0, \; 2^1, \; 2^2, \; 2^3, \; 2^4, \;\dots
                                                                             \]
                                                                             Eine \emph{brennende} Kerze bedeutet ``+diesen Wert'', eine \emph{nicht brennende} Kerze bedeutet ``+0''.

                                                                             \subsubsection*{Fragen aus dem Text}
                                                                             \begin{enumerate}
                                                                               \item \textbf{An welchen Geburtstagen kommt eine weitere Kerze dazu?} \\
                                                                                 Genau dann, wenn eine neue Zweierpotenz erreicht wird: 1, 2, 4, 8, 16, 32, \dots

                                                                                   \item \textbf{An welchen Geburtstagen müssen alle vorhandenen Kerzen brennen, an welchen nur eine?}
                                                                                     \begin{itemize}
                                                                                         \item \emph{Alle vorhandenen Kerzen brennen}, wenn das Alter eine weniger als die nächste Zweierpotenz ist, z.\,B. 3 (nächste Potenz 4), 7 (nächste Potenz 8), 15 (nächste Potenz 16) usw.
                                                                                             \item \emph{Nur eine Kerze brennt} bei den Zweierpotenzen selbst: 1, 2, 4, 8, 16, \dots
                                                                                               \end{itemize}

                                                                                                 \item \textbf{Welche Anzahl von Kerzen wird auch im hohen Alter ausreichend sein?} \\
                                                                                                   Man benötigt so viele Kerzen, dass \(\,2^{\text{(Anzahl Kerzen)}} - 1\) größer (oder gleich) dem maximalen Alter ist. Beispiel: Um bis 127 Jahre abdecken zu können, braucht man 7 Kerzen (\(2^7 - 1 = 127\)).
                                                                                                   \end{enumerate}

                                                                                                   \subsubsection*{Woran erkennt man, wer älter ist?}
                                                                                                   Es ist erkennbar an den \emph{höheren Kerzen}, also den größeren Zweierpotenzen. Wer z.\,B.\ schon die Kerze für \(2^4 = 16\) brennen hat, ist mindestens 16 Jahre alt; jemand ohne diese Kerze ist jünger.

                                                                                                   \subsubsection*{Allgemeine Regel}
                                                                                                   \emph{Jedes Alter wird als Summe von Zweierpotenzen dargestellt. Jede Kerze für \(2^i\) ist entweder an (wenn diese Potenz im Alter enthalten ist) oder aus. So lässt sich eindeutig und schnell erkennen, wie viel jemand ``auf dem Kuchen'' an Lebensjahren hat, ohne viele Einzalkerzen.}

                                                                                                   \bigskip
                                                                                                   \textbf{Zusammenfassung:}
                                                                                                   \begin{itemize}
                                                                                                    \item \textbf{Aufgabe 1:} Binärsystem, 4 Lampen für Werte 0--15. Für 32 und 47 müssen wir auf 6 Lampen erweitern (32 = \(100000_2\), 47 = \(101111_2\)).
                                                                                                     \item \textbf{Aufgabe 2:} Eigenes Merkblatt zur Binärdarstellung erstellen (Potenzen von 2, Umwandlung, Beispiele).
                                                                                                      \item \textbf{Aufgabe 3:} Geburtstags-Kerzensystem als Binärsystem; neue Kerze bei jeder Zweierpotenz, man erkennt höheres Alter an höheren Zweierpotenzen. Wenige Kerzen reichen für große Zahlen.
                                                                                                      \end{itemize}

                                                                                                      \end{document}
                                                                                                      